\documentclass[conference]{IEEEtran}
\IEEEoverridecommandlockouts
\usepackage{cite}
\usepackage{amsmath,amssymb,amsfonts}
\usepackage{algorithmic}
\usepackage{graphicx}
\usepackage{textcomp}
\usepackage{xcolor}


\begin{document}

\title{O problema da cultura de doações e uma proposta de solução usando Sistema de Recomendação\\
}

\author{\IEEEauthorblockN{1\textsuperscript{st} Gabriel Silva de Castro}
\IEEEauthorblockA{\textit{Campus Sul Edson Machado} \\
\textit{Instituto de Educação Superior de Brasília}\\
Brasília, Brasil \\
gabrielcastrog2@gmail.com}
\and
\IEEEauthorblockN{2\textsuperscript{nd} Gabriel Prado Silva}
\IEEEauthorblockA{\textit{Campus Sul Edson Machado} \\
\textit{Instituto de Educação Superior de Brasília}\\
Brasília, Brasil \\
gaprados3@gmail.com}
\and
\IEEEauthorblockN{3\textsuperscript{rd} Guilherme Araújo de Castro}
\IEEEauthorblockA{\textit{Campus Sul Edson Machado} \\
\textit{Instituto de Educação Superior de Brasília}\\
Brasília, Brasil \\
guilhermearaujo3112@gmail.com}
\and
\IEEEauthorblockN{4\textsuperscript{th} Luiz Eduardo de Almeida Toledo Leal}
\IEEEauthorblockA{\textit{Campus Sul Edson Machado} \\
\textit{Instituto de Educação Superior de Brasília}\\
Brasília, Brasil \\
luiztoledo@id.uff.br}
}

\maketitle

\begin{abstract}
Instituições não governamentais pertencentes ao Terceiro Setor brasileiro vem sofrendo com a redução de doações nos últimos anos, principalmente devido a recente pandemia global, onde provocou aumento significativo nos níveis de pobreza. Esse acontecimento ocasionou aumento da necessidade de doações direcionadas a ONGs. Entretanto, no cenário atual de distanciamento das instituições com a população brasileira, o alcance de ONGs vem diminuindo cada vez mais. Nesse aspecto, a tecnologia pode ser de grande ajuda as instituições não governamentais, tanto para propor relação mais próxima com doadores ou possíveis doadores, quanto para aumentar a organização de instituições e seus projetos sociais. Diante disso, este trabalho tem como objetivo se aliar a tecnologia para desenvolver uma solução que aumente o alcance de ONGs e as conecte com doadores e interessados em se tornar doadores. Para isso, será construído um aplicativo que utiliza Sistema de Recomendação e recomenda ONGs similares ao perfil de interesse do usuário.
\end{abstract}

\begin{IEEEkeywords}
ONGs, Terceiro Setor, cultura de doações, Sistema de Recomendação e causas sociais.
\end{IEEEkeywords}

\section{Introdução}
O empobrecimento da população se tornou um problema ainda maior nos últimos anos, a sociedade sofreu mudanças drásticas ao enfrentar uma pandemia global, sobretudo na parcela mais pobre da população, que consequentemente gerou mais pessoas em situação de rua (SILVA, 2020). Segundo a pesquisa feita pela Rede Penssan (Rede Brasileira de Pesquisa em Soberania e Segurança Alimentar e Nutricional) em 2021, a fome atingiu 9\% da população brasileira, totalizando 19,1 milhões de pessoas com fome em 2020. Diante de situações graves como uma crise humanitária, o Terceiro Setor é fundamental para auxiliar pessoas carentes e em situação de rua.

No Terceiro Setor existem organizações capacitadas para auxiliar a população em diversas causas, como combate à fome e à pobreza, crianças carentes, causas ambientais e pessoas em situação de rua. Porém, organizações sem fins lucrativos necessitam de recursos para cumprir seus objetivos, recursos estes que são arrecadados em doações. Depender de doações é um problema que faz com que ONGs não tenham estabilidade financeira para suprir necessidades de forma recorrente. 

A cultura de doações no Brasil ainda é fortemente influenciada por questões políticas e sociais, segundo o Brazil Giving Report, realizada pela CAF (Charities Aid Foundation), grande parte de doações são direcionadas a causas religiosas com cerca de quase metade das doações (49\%). Outras causas como apoio a crianças e jovens (39\%) e combate à pobreza (30\%) são também incluídas na lista, mas com porcentagens menores. 

No Brasil, apesar da grande desigualdade social e do empobrecimento da população, pesquisas mostram que existe uma parcela da população que deseja contribuir com doações, mas não sabe como ou tem dúvidas na procura de uma organização. 

Além da falta de proximidade de ONGs com o povo brasileiro, existe uma certa desconfiança de uma parcela da população com o uso do dinheiro doado. As perspectivas de não doadores em relação a instituições não governamentais foram pesquisadas por Micheli Umebayashi (2018), e em sua pesquisa doze entre quinze entrevistados declaram ter percepção negativa das ONGs no Brasil, indicando mau uso do dinheiro ou corrupção e lavagem de dinheiro. Ainda em sua pesquisa, a autora descreve que a relação entre doadores e ONGs começa com a presença de estímulos internos ou externos que marcam a vida de um indivíduo e o influencia a procurar e doar para causas sociais. \cite{umebayashi2018doaccoes}




\section{Contextualização}
O Brasil é um país em que a cultura de doações não é bem divulgada e nem incentivada. Segundo o World Giving Index (WGI), o Brasil é o 54º país mais generoso do mundo. No início esse dado é um pouco estranho, pois quando se pensa em Brasil, imagina-se um povo solidário, e de fato grande parcela da população brasileira deseja realizar doações, entretanto, não tem informação de onde e como encontrar ONGs e nem tempo suficiente para se disponibilizar em trabalhos voluntários. \cite{sitepolitize01}

De acordo com Yago Silva (2020), o Terceiro Setor enfrenta muitos problemas em relação ao Marketing Social presente nas organizações. Em seu estudo, foi identificado que as entidades sociais não possuem profissionais especializados em Marketing, comprometendo a aquisição de novas doações para as instituições. Além disso, estudos anteriores comprovam que profissionais de diversos setores são necessários para manutenção das instituições no Terceiro Setor. \cite{silva2020dificuldades}

A tecnologia está presente na vida da maior da parte da população, e a tecnologia pode ser uma grande aliada do terceiro setor brasileiro no objetivo de ampliar as fontes de recursos e doações das organizações. A digitalização das instituições tem o poder de transformar o Terceiro Setor, utilizar tecnologia como intermediária para doações e divulgação de projetos sociais pode agilizar a administração de recursos e organizar a divulgação das mais diversas causas sociais.  

\subsection{Digitalização das Instituições}
As necessidades das instituições do Terceiro Setor sempre são urgentes, pois são instituições que atuam para o desenvolvimento social. No entanto, o alcance dessas instituições é limitado, criando barreiras na busca de doações e voluntários.\cite{sitecodebit}

Na busca de maneiras para quebrar as barreiras da difícil aproximação entre instituições não governamentais e possíveis doares, a modernização e digitalização das ONGs pode ser a solução de diversos problemas.

A partir da digitalização, as instituições são capazes de divulgar seus projetos sociais em amplas plataformas, e até mesmo, encontrar e engajar novos públicos que não são familiarizados com a cultura de doação. 



\section{Problema}
A dificuldade de acesso à informação sobre campanhas de doação e incentivo a solidariedade é um grande problema para ONGs. Algumas causas sociais chamam mais atenção da população como crianças carentes e combate à pobreza, porém, outras causas como questões ambientais e luta por direitos humanos passam despercebidas. Isso ocorre por não serem causas amplamente divulgadas e visíveis no cotidiano, o indivíduo se sensibiliza mais pelo que é visto. 

Cada ONG possui uma ou várias causas, cada causa tem seu objetivo e cumpre um papel social importantíssimo. Procurar e achar organizações que promovem causas sociais específicas pode ser trabalhoso, além disso, acompanhar suas ações sociais consome tempo, e a falta de tempo livre é um dos fatores que impede brasileiros de engajar em causas sociais. Falta de mobilidade para trabalhos voluntários em ONGs também é um dos motivos para o desinteresse na cultura da doação. \cite{goff2020identifying}

A desinformação somada a falta de tempo gera ainda mais desinteresse da população na cultura de doações. Pensando nesse cenário com diversos problemas, como a tecnologia pode ser uma aliada das ONGs e ajudar no desenvolvimento da cultura da doação?





\section{Objetivo Geral}
Devido a existência de grande quantidade de ONGs atualmente, o possível doador pode encontrar dificuldades em achar uma instituição que esteja alinhada aos seus interesses. Diante disso, este trabalho tem como objetivo desenvolver um aplicativo para promover a facilidade em doações a ONGs, e através do uso de Sistema de Recomendação, sugerir instituições similares que necessitam de doações, e assim, incentivando a cultura de doação nos mais diversos tipos de causas sociais.

\subsection{Sistema de Recomendação}
Os Sistemas de Recomendações são mecanismos que oferecem técnicas para fornecer sugestões de itens ao usuário. São capazes de sugerir desde músicas, vídeos e até produtos. Neste trabalho, será usado Sistema de Recomendação para sugestão de ONGs com causas sociais que tenha objetivos em comum com interesses do usuário. A técnica utilizada será a recomendação baseada em conteúdo. 

\subsubsection{Recomendação baseada em conteúdo}
\hbadness=99999 Sistemas de recomendação baseada em conteúdo recomendam itens levando em consideração a descrição do item e o perfil do usuário \cite{sharma2013survey}. Tipicamente, o perfil de usuário é criado automaticamente se baseando no que é consumido, entretanto, neste projeto o perfil será criado a partir do que o usuário decidir como descrição de interesse. 

\subsubsection{Outras técnicas}
Além da recomendação baseada em conteúdo, existem outras duas técnicas de recomendação. 
A primeira técnica é a recomendação com filtragem colaborativa, nela a semelhança de itens é medida pelo feedback do usuário, geralmente por meio de avaliações. A relação entre opiniões de diferentes usuários é explorada, assumindo que usuários com avaliações similares terão o mesmo gosto no futuro. \cite{sharma2013survey}

A terceira e última técnica é o processo hibrido, combinando outras duas técnicas e reduzindo suas desvantagens. Entretanto, o resultado da técnica hibrida depende do domínio e dos dados. \cite{ccano2017hybrid}


\section{Objetivos Específicos}
\begin{itemize}
\item Facilitar o acesso a doações para ONGs para que se torne uma atividade simples e que não consuma tempo;
\item Utilizar celulares como intermediador entre ONGs e doadores;
\item Incentivar a cultura de doações;
\item Aumentar o interesse em diferentes causas sociais, principalmente causas invisibilizadas.
\end{itemize}



\section{Referencial Teórico}
\subsection{Terceiro Setor: seus objetivos e dificuldades}
A definição do Terceiro Setor vem de instituições que não são do setor do Estado e nem do setor privado, são organizações formadas sem fins lucrativos e para prestar serviços públicos. \cite{sitepolitize02} A existência do Terceiro Setor é baseada na sua capacidade de atender demandas sociais que o Estado não consegue suprir. As instituições não governamentais vem tentando agir através do auxílio às pessoas carentes. \cite{silva2020dificuldades} Contudo, o Terceiro Setor enfrenta dificuldades, principalmente após um cenário pós-pandemia.
A deficiência de recursos das ONGs é um dos principais problemas. A falta de planejamento por parte de ONGs, aliada ao desinteresse da população na cultura de doação, constituiu grande falta recursos nos últimos anos. Cerca de 41\% das instituições carecem de apoiadores financeiros, 13\% de doação material como alimentos e equipamentos e 11\% das instituições precisam de voluntários. (Datafolha, 2020)


\subsection{A digitalização de ONGs}
Diante dos variados problemas nos últimos anos, principalmente durante a pandemia do Coronavírus, diversas ONGs se viram obrigadas a procurar meios na tecnologia para administrar seus meios de arrecadação de recursos. Segundo um Estado feito pelo Fórum Econômico Mundial em 2021, organizações que se reinventaram digitalmente conseguem se relacionar com o público mais profundamente. \cite{siteimpacto} \cite{economicforum}

\section{Trabalhos Correlatos}

\subsection{Sistema de Recomendação de Artigos Científicos utilizando dados sociais}
No artigo citado, o autor compara a comunidade científica com uma rede social, devido as relações encontradas nela, como coautoria,
colaboração em projetos, orientações, dentre outros. Por conta da
dificuldade que alguns pesquisadores enfrentam ao buscar novos
artigos para estudo, o autor propõe um sistema de recomendação de
artigos, utilizando informações sociais e bibliométricas, comparando
os algoritmos propostos pelo projeto com algoritmos já consolidados
na área de sistemas de recomendação, e com algoritmos encontrados
em trabalhos correlatos. \cite{grava2021sistema}

\subsection{Sistema de Recomendação baseado em conteúdo textual: avaliação e comparação}
No artigo acima, o autor propõe uma metodologia para comparação de
algoritmos de recomendação que vai além da precisão das previsões, e
a sua utilização para avaliar sistemas de recomendação baseados em
Conteúdo Textual e Filtragem Colaborativa, na tarefa do tipo
"recomendar bons itens". \cite{silva2017sistema}

\subsection{Sistema de Recomendação Sensível ao Contexto para atividades de lazer}
Neste artigo o autor propões um modelo de recomendação baseado nas abordagens clássicas da teoria de recomendação, como a filtragem colaborativa, adicionando informações contextuais do usuário, para não só limitar as recomendações, mas também incorporar em um sistema de pontuação essas informações do usuário. \cite{avancini2016sistema}


\section{Resultados Esperados}
A partir da pesquisa e do desenvolvimento do aplicativo, espera-se contribuir com a cultura da doação por meio da facilitação e divulgação de causas sociais através de recomendações de ONGs similares ao histórico de doações do usuário. Com isso, espera-se o aumento de doações feitas a causas sociais, principalmente causas invisibilizadas ou pouco divulgadas e expanda o alcance de ONGs pouco procuradas.



\section{Cronograma}
\begin{itemize}
\item Pesquisas e leitura de artigos;
\item Prototipagem da aplicação do sistema;
\item Estruturação do algoritmo de recomendações;
\item Desenvolvimento do sistema para plataforma mobile, utilizando Flutter.
\end{itemize}


\section{Bibliografia}
\bibliographystyle{ieeetr}
\bibliography{Bibliografia}


\end{document}
